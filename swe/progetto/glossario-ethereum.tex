\documentclass{book}
\usepackage[utf8]{inputenc}

\title{Glossario Ethereum}
\author{Federico Bicciato}

\begin{document}
	\maketitle
	
	%	TO DO
	%	Consultare in ordine:
	%		foglio del capitolato C6
	%		Wikipedia: Ethereum, blockchain, smart contracts, solidity
	%		link forniti da RedBabel
	%		Mastering Ethereum
	%
	%	SRA: 
	
	\section{Glossario}
		\paragraph{Dapp}
			Applicazione decentralizzata. Evita di interagire con intermediari, come accade nell'architettura client-server. Ogni nodo nella rete sa fare il suo lavoro in modo indipendente, senza dipendere dagli altri. Usano degli smart contract per fare calcoli in EVM.
		\paragraph{account}
			Gli account sono oggetti presenti nella rete di Ethereum. Comunicano tra loro, eseguono codice, tengono un database interno.
		\paragraph{ETH}
			sigla per Ether
		\paragraph{Ether}
			Criptovaluta generata tramite la piattaforma Ethereum. Si paga per acquistare Gas.
		\paragraph{Ethereum}
			La piattaforma per implementare smart contracts. Ha le seguenti caratteristiche:
			\begin{itemize}
				\item open source
				\item pubblica
				\item basata su blockchain
				\item distribuita: ogni nodo fa i suoi calcoli
				\item fa uso di smart contracts
			\end{itemize}
		\paragraph{EVM} sigla per Ethereum Virtual Machine
		\paragraph{Ethereum Virtual Machine}
			E` il nome della parte di Ethereum che gestisce, nella rete, lo stato interno e la computazione. I numerosissimi oggetti contenuti nella rete sono gli account.\newline EVM assicura che il codice venga eseguito ugualmente in tutte le parti della rete.
		\paragraph{Gas}
			Il carburante che fa funzionare Ethereum. Si spende per fare operazioni. Si paga in valuta ETH.
		\paragraph{smart contract}
			Contengono codice. I nodi che li eseguono verificano l'integrita` dei contratti e dei loro output.
\end{document}